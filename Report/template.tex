\documentclass[a4paper,12pt]{article}
\usepackage{float}

%% Language and font encodings
\usepackage[english]{babel}
\usepackage[utf8x]{inputenc}
\usepackage[T1]{fontenc}
\usepackage{amsmath}
\usepackage{graphicx}
\usepackage[colorinlistoftodos]{todonotes}
%% Sets page size and margins
\usepackage[a4paper,top=3cm,bottom=2cm,left=3.5cm,right=3.0cm,marginparwidth=1.75cm]{geometry}

%% Useful packages
\usepackage{amsmath}
\usepackage{graphicx}
\usepackage[colorinlistoftodos]{todonotes}
\usepackage[colorlinks=true, allcolors=blue]{hyperref}
\usepackage{multibib}


\begin{document}


\begin{titlepage}

\newcommand{\HRule}{\rule{\linewidth}{0.5mm}} 
\center % Center everything on the page
 
%---------------------
%	HEADING SECTIONS
%---------------------

\includegraphics[width=10cm]{epfl_logo.png}\\[1cm]

%---------------------
%	TITLE SECTION
%---------------------

\HRule \\[0.4cm]
{ \huge \bfseries Semester Project Report}\\[0.4cm]
{ \Large \bfseries Toward a generic open source $C_{++}$ reflex model for the design of exoskeleton controllers~:\\[0.3cm]an application to the Hibso device}\\[0.4cm]
\HRule \\[1.5cm]
 
%--------------------
%	AUTHOR SECTION
%--------------------

\begin{minipage}{0.4\textwidth}
\begin{flushleft} \large
\emph{Author:}\\
Jean \textsc{Gschwind} % Your name
\end{flushleft}
\end{minipage}
~
\begin{minipage}{0.4\textwidth}
\begin{flushright} \large
\emph{Professor:} \\
Auke Jan \textsc{Ijspeert}\\
\emph{Supervisor:}\\ % Supervisor's Name
Florin \textsc{Dzeladini} 
\end{flushright}
\end{minipage}\\[2cm]

%------------------
%	DATE SECTION
%------------------

{\large \today}\\[2cm]

\vfill

\end{titlepage}

\newpage
\tableofcontents
\newpage

%------------------
%	INTRODUCTION
%------------------

\section{Introduction}
\medbreak
The goal of this project is to test a new control strategy on an existing exoskeleton. The main idea is to use the torque generated by a bio-inspired walking model as motor command. The bio-inspired model is based on the reflex-based walking model of H.Geyer~\cite{geyer2010muscle}. The idea of using bio-inspired controller as reference torque command is a new paradigm in the control of exoskeleton. This paradigm has already been tested within the Symbitron European project and the first results are promising. The interesting aspects of the reflex model are that the output of the controller is not a fixed reference pattern. Instead, the pattern emerges as an interaction between the robotic device, the human in the loop and the environment.
\medbreak
The underlying hypothesis is that the brain relies on low level muscle reflex loops. The upper brain relies on those low level control mechanism which greatly simplifies the control. The other interesting aspects of the proposed controller are that it uses simulated muscles which exhibits interesting spring and damper properties which could help create better exoskeleton controller.
\medbreak
Compared to other controllers, this model does not use any reference pattern. Instead, the pattern emerges as an interaction with the environment. It is important because the model does not force any trajectory like other models. In this perspective, the subject can start any movement he wants : the controllers will not work against him. With this in mind, used motors need a high degree of transparency, passive (when the subject begin its movement) and active (when the walking is started) in order to help and not interfere. The different angles and the reflex loops signals are used to compute the force generated by each muscle. The forces generated by the muscles acting on a given joint are then combined to produce a given torque used as a motor command.
\medbreak
With this project, we will see an application of the H.Geyer theory~\cite{geyer2010muscle} over a control of an hip orthosis called Hibso. The possibilities for disabled people or for reeducation can be fantastic. However, some new questions appear if it is the case : if we just control one muscle (and all other move according to this one), how does the patient respond to the interaction with the exoskeleton in term of walking efficiency or in term of comfort ? And what is the effect on the energy consumption ?
\medbreak
The main goals of this project are :
\begin{enumerate}
\item Adapt the \textit{Libnmm} library for utilization in real-time devices.
This mainly includes the optimization of the main control loop duration to fulfill the real time constraints of the device.
\item Chose and adapt the reflex loops based on available sensors and motors.
\item Create an efficient and usable controller for the Hibso device
\end{enumerate}

\newpage

\section{Theory: from human walking simulation to exoskeleton controller}
\subsection{Introduction: reflex based model of walking}

The  central pattern generator (CPG) is a neural network localized in the spinal cord~\cite{dimitrijevic1998evidence}. Present in almost every vertebrate, this network produces a rhythmic and static pattern to create movement. Some scientists proved~\cite{ijspeert1999evolution} that the CPG lamprey, by artificially stimulating the spine, does not need external sensors to operate and therefore a high level (brain) control and reflex loops does not have influence in lamprey locomotion. For humans, the CPG is extremely more complex and studying the vertebral column of the lamprey is impossible. However, some articles~\cite{dzeladini2014contribution} suggest the presence of other components for the locomotion like the reflex loops.
\medbreak	
The H.Geyer's model~\cite{geyer2010muscle} is based on the use of reflex loops. But what is a reflex loop ? The key to understand this concept is the following. The legged movement is not only controlled by the CPG but also by self-induced and autonomous reflexes. Muscles apply some constraints on other muscles in reaction on disturbances and this without any brain contribution. This reflex are correlated to the gait cycle. Some studies~\cite{pearson2006assessing} show that reflex loops seem more important than CPG for human locomotion. 
\medbreak
One of the most important information needed by this model is the ground contact : if the foot is \textit{stance} or \textit{swing}. It is important because during each step, the foot works in an opposite way. During stance, the extensor muscles have more activity to support weight bearing. While during swing, the flexor muscles act in order to bring the leg in front.
\medbreak
Since the control of muscle to generate locomotion emerges from itself, the main benefit of this model is that it can easily adapt to ground disturbances without any control of an operator.

\begin{figure}[H]
	\centering
    \includegraphics[width=11cm]{geyer_2.JPG}
    \label{geyer_2}
    \caption{Reflex loops example. Taken from~\cite{geyer2010muscle}. On (B), one leg is \textit{stance} and the other is \textit{swing}. On the swing part, you can see the feedback of the soleus muscle (SOL) and the lumped vasti group muscles (VAS). The other pictures show the evolution of the feedback on each muscles on each foot.}
\end{figure}

\subsection{Controller design} \label{controller_design}
%ADDED
The idea behind this project is based on the hypothesis that using a bio-inspired model - which reproduces both external (CoM positoin, ground reaction forces) as well as internal (muscle activities) observed during human walking - as an exoskeleton controller will better interface with the human body. 

In our case, since we want to control the Hibso device, we do not use the entire H.Geyer model~\cite{geyer2010muscle}. The available sensors and actuators forces us to use a simplified version of this model. 
\label{controllerdesign}
\medbreak
The exoskeleton prototype only actuates the sagital hip motion. Therefore, only the the gluteus muscle group (GLU), the hip flexor muscle group (HFL) and the hamstring (HAM) can be used (as they are the only muscles in the model which generate forces around the hip joint). In this case we do not use the HAM because it is a bi-articular muscle and needs the angular information of both the knee and the hip to fully know its state.
%The prototype of the available exoskeleton only contains the hip part. Moreover, we only keep the gluteus muscle group (GLU) and the hip flexor muscle group (HFL) reflex loops. The hamstring (HAM) is a bi-articular muscle, it acts on both the hip and the knee. Here, we choose not to use. % it since we might disturb the patient with an additional influence of the knee reflex loop since we do not control it.
\label{HAM}
\medbreak
We can extract four important modules from the H. Geyer Reflex model (see Table~\ref{modules} 
%in order to know what sensors are needed to achieve each module) :

\begin{enumerate}
	\item Trunk Stabilization (figure~\ref{geyer_2}, picture (D)). This feedback uses a PD controller adapted to work with muscles that ensures that the trunk stays in a given position (with respect to gravity). This implies that a way to get the direction of gravity is needed, such as an IMU. For now, we don't use it because the IMU necessary is messing.
    \item Push off  (figure~\ref{geyer_2}, picture (B and C)). The foot will rise up. The biggest part of the reflex feedback is from the low part of leg (knee and ankle). Actually, we do not use this model because of the sensor and motors lack.
    \item Swing  (figure~\ref{geyer_2}, picture (F)). The foot swing to his new position. Despite the feedback number, we only simulate the force form the GLU and the HFL intended to create the rotation movement.
    \item Weight Transfer (figure~\ref{geyer_2}, picture (E)). When a human is walking, there is one moment the two feet are in stance, this is called the double stance phase. At this moment, the weight is transferred from one foot to another. This results in a decrease of the muscles extension in the leg preparing for swinging and an increase in extensor activity in the leg entering in stance. On the Geyer model, the flexion/extension increase/decrease in function of the weight under the foot. In our case, we do not have an exact information about weight, see section~\ref{soilintegration}. Therefore, we use a constant value which is added or removed like the model extension inspired from J.Wang~\cite{wang2012opti}.
\end{enumerate}

The Table~\ref{modules} shows the four different modules, the associated gait phase on which they act, the actuated joints and the needed sensors.

% UNE NOUVELLE COLONNE A ETE AJOUTEE AU TABLEAU
% Le caption est plus detaillé.
\begin{table}
  \centering
  \begin{tabular}{c|c|c|c|c|c}
    Modules & St/Sw & Hip & Knee & Ankle & Sensor needed \\[0.01cm]
    \hline
    Trunk Stabilization  & St & X & & & Gyroscope \\
    Push off & St &  & X & X & Angular information \\
    Leg placement & Sw  & X & X & X & Angular information \\
    Weight Transfer & St & X & & & Vertical GRF*
    \label{modules}
  \end{tabular}
    \caption{Decomposition of the reflex model into 4 different modules. For each modules we show the phase on which the module is active, the joint on which it acts and the sensor needed. The Weight transfer modules as been adapted to use only the contact information instead of the ground reaction forces, see text for details.}
    %\caption{Sensors needed to achieve each module }
\end{table}
     
\subsection{Controller assessment}
\label{controllerassessment}

Soil changes, sudden speed changes or more generally external disturbances are usually the weak point of an exoskeleton. 
However, as this controller is based on a bio-inspired and reflexive model we expect higher robustness to this kind of problems. We also expect that if the model used reproduces well enough the real dynamics of walking, the subject should less feel the device than with a traditional controller. The prototype and its use give us the chance to confirm or infirm many important points:
%TU PEUX PAS DIRE by the way, :) However, the this new model theory is more robust to this kind of problem. By the way, the patient need to feel the the device as less as possible , even after a long duration. The prototype and its use give us the chance to confirm or infirm many important points :

\begin{enumerate}
	\item Energy consumption. A long usage involves an important battery and/or a small energy consumption. A big battery is obviously not a good solution for the patient's comfort and for the price. The new controller may be more efficient in terms of energy consumption because in the ideal case, the exoskeleton should work together with the patient. A more classical controller wants to apply a pattern all the time, if the subject diverge, even slightly, from the initial pattern, the motors will work against the muscles and cause an increase in consumption. %This controller does not have any pre-established trajectory so the motors will never work counter the patient. 
    This controller does not have any pre-established trajectory and therefore will not over-constrain the motion of the leg. 
    \item EMG. It needs additional material. However, it is important to study the patient muscles response to the assistance because the motors needs to work in cooperation with the muscles. A muscle activity decrease means that the controller works in fit with the patient. A muscle activity increase means that the controller does not allow enough freedom to the patient (and it will increase the energy consumption, as stated before).
    \item Walking stability. The controller must ensure a perfect stability to the patient. The exoskeleton itself should never cause a fall.
    \item Patient's comfort. If the exoskeleton does not work perfectly with the patient, it will not only cause an energy consumption increase but will also affects the patient experience. Furthermore, the current controller (described more precisely in~\ref{actualcontroller}) needs a little time to learn the walking rhythm. If we adjust speed, it need again a little time to adapt his frequency. This delay creates discomfort. With the new controller, we do not suppose having these issues anymore. The tests will be a good opportunity to verify this hypothesis.
    \item Patient's exhaustion. The goal of the assistance is to relieve the patient and not cause additional tiredness. This variable can also easily check by comparing the exhaustion between this controller and the old one. Once again the energy consumption, the patient's comfort and the patient's exhaustion are linked. Indeed, if the exoskeleton works against the patient wish, it will cause additional exhaustion, which is obviously not wanted.
\end{enumerate}

\subsection{Scientific perspective}

An obvious interesting scientific perspective of this project is to improve rehabilitation procedures and facilitates patients life. But could we use this approach to infer some results on the underlying neural architecture involved during locomotion ? As discussed in Section~\ref{controller_design} we decided to not use the HAM muscles because we don't have knee angular information, which is needed for the calculation of the muscle length. But if we manage somehow to calculate the muscle length (by either, calculating the knee angle, or estimating it), what would be the effect of using only the HAM muscles contribution to the HIP ?
%In our case, we only use mono articular muscles (GLU and HFL) by obligation. How does it affect the movement ? We may have to control one motors and every other will react in consequences. For example, the two articular vasti muscle group (VAS) is connected to the hip and the knee. In our case, the VAS feedback will only act on the hip. What about the result ?
\medbreak
%A patient important discomfort would mean that the knee link are very important and we cannot ignore this component. If this is the case, it suggests that this reflex loops is important in locomotion because the patient cannot walk properly without this feedback.
A patient important discomfort would mean that this specific reflex loops is important in locomotion  because the patient cannot walk properly with only a partial feedback.

Conversely if the patient does not see any difference, it may mean that the consideration of the reflex loops are maybe overvalued. 
\medbreak
In conclusion, an underlying side goal of the project and the approach is the possible validation of certain reflex loops : are the loop reflexes as important as H.Geyer assumes ? 
If at the end, we have good results, it will not confirm or infirm the hypothesis. It will rather mean that we are on the good way to deeper understand human locomotion.
If the results are negative (see section~\ref{controllerassessment} for more explanations on the controller assessment), it may mean humans do not use this kind of system like simpler animals. Maybe the locomotion is only manageable by upper brain neural structure.
\medbreak
Ideally, this will give more information about how human locomotion works : is it only controlled by the CGP, is it a mix of CPG actions and reflex loops and in what proportion or are every components totally controlled by a very high and central level control (the brain) ?

%\subsection{Optimization}
\subsection{Model detail and optimization}

The model is composed of two parts: the body, modeled as a rigid body with 5 segments and 6 joints, and bio-inspired control architecture. The bio-inspired control is generated by muscles whose activities is generated by different reflex loops (see Figure~\ref{geyer_2}. The ground reaction forces, the muscle length and the muscle force are the sensors from which the reflex loops are derived. 
%In the model, human muscles have two inputs : activation and the joints angles on which they act. The activation comes from the reflex loops connected to the muscles as shown in figure~\ref{geyer_2}. The angles comes from motor sensors. 
\medbreak
%CHANGED : For some simplification reasons, muscles are considered as a combination of spring and damper system. They are all composed by four components :
The muscles are modeled together with their tendon to form a muscle tendon unit, see Figure~\ref{muscletendonmodel} for a schematic. 
% PUT INTO Figure caption 


\begin{figure}[H]
	\centering
    \label{muscletendonmodel}
	\includegraphics[width=10cm]{tendon_model.JPG}
    \caption{ H.Geyer muscle model. The muscle model is made of two systems (the muscle and the tendon) connected in series. CE simulates the muscle contraction potential, BE models the collapsing prevention of the muscle if SE is too slack, SE models the muscle elastic potential, PE simulates additional elastic potential when the CE stretches beyond a certain value. 
    The length lce and the force applied at the SE segment FSE are the source of muscle sensors.}
\end{figure}

%The reflex parameters values proposed in~\cite{geyer2010muscle} by H.Geyer could not simply be resumed and used in \textit{Webots} simulation (more explanation about \textit{Webots} in part~\ref{simulation}). Previous work in the Biorob laboratory has been done to adapt the initial H.Geyer's Simulink simulation to a c and c++ model working on \textit{Webots}. This work has also optimized parameters for different speed and the addition of CPG influence with the help of the particle swarm optimization (PSO). This method finds the most efficient parameters to stabilize walking the fastest way possible, reduces the energy consumption and increases the distance traveled.
Previous work in the Biorob laboratory have been done to adapt the initial H.Geyer's Simulink simulation to a C and then C++ model working on \textit{Webots}. The reflex parameters values proposed in~\cite{geyer2010muscle} by H.Geyer would not generate a stable locomotion when used in the \textit{Webots} simulation (more explanation about \textit{Webots} in part~\ref{simulation}). Working parameters were obtained with the help of the particle swarm optimization (PSO). The fitness function used was the minimization of the cost of transport (maximizing distance, while minimizing the energy consumed).

% The parameter used in this work were re-optimized on a version of the model that did not had the need of ground reaction forces.

\newpage*
\section{Material and methods}

\subsection{Hardware}\label{hardware}

The hardware is composed of three parts :
\begin{enumerate}
  \item \textit{Beagle Bone Black}~\cite{beagleboneblack} (figure~\ref{beagleBone}) is a micro controller produced by the BeagleBoard.org Foundation. The card needs to be powered with a 12-24 continue current. Equipped with AM335x 1GHz ARM® Cortex-A8 processor, it runs on Debian "Jessie" version. We can connect the card with wifi or with an ethernet cable (The procedure to follow to connect in different ways is explained in the Walki manual file \cite{walkisoftwarepack}).The processor does not allow compiling the library or the HibSo Reflex controller itself. We thus need to do a cross compilation. The library is cross compiled and sent to the card using the script 'install' with \textit{BeagleBone} IP address as an argument. The HibSo Reflex code is cross compiled with Qt (the procedure is also explained in the manual \cite{walkisoftwarepack}).
  \item Motorboard (figure~\ref{motorboard}) is the sensor used for the joint angle and the motor controller. It is a self made card designed by Romain Baud. Note that the use card for the tests is not the most recent card available.
  \item Self made card, plugged on the \textit{Beagle Bone}, designed by Romain Baud and intended to have all the necessary sensor connections like \textit{ads7844} (figure~\ref{ads7844}) (a 8-Channel Serial Output Sampling Analog-To-Digital Converter produced by Texas Instrument, used for the communication with the soles) or the motorboard (to recover the leg angle values and control the motors). 
  \item Smart Foot Sensor~\cite{soles} (figure~\ref{sole}). Composed of eight cells 
\end{enumerate}

\begin{figure}[H]
\begin{minipage}[c]{.46\linewidth}
     \begin{center}
             \includegraphics[width=7.6cm]{beaglebone.jpg}
             \caption{\textit{BeagleBone}}
             \label{beagleBone}
         \end{center}
   \end{minipage} \hfill
   \begin{minipage}[c]{.46\linewidth}
    \begin{center}
            \includegraphics[width=7.6cm]{ars.jpg}
            \caption{ads7844}
            \label{ads7844}
        \end{center}
 \end{minipage}
\end{figure}

\begin{figure}[H]
\begin{minipage}[c]{.46\linewidth}
     \begin{center}
             \includegraphics[width=7.6cm]{motorbord.jpg}
             \caption{Motorboard}
             \label{motorboard}
         \end{center}
   \end{minipage} \hfill
   \begin{minipage}[c]{.46\linewidth}
    \begin{center}
            \includegraphics[width=7.6cm]{sole.jpg}
            \caption{Smart Foot Sensor~\cite{soles}}
            \label{sole}
        \end{center}
 \end{minipage}
\end{figure}


\subsection{Software}

In this project, we use many previous works from Biorob and LSRO Laboratory, like the \textit{Walki Software Pack} or the library \textit{Libnmm}. You can see a more visual idea from the project architecture below (each part is described more precisely below) :

\begin{center}
\includegraphics[width=0.8\textwidth]{sch_ma.JPG}
\end{center}

\subsubsection{Walki Software Pack}

The \textit{Walki Software Pack}~\cite{walkisoftwarepack} is developed by Romain Baud and is dedicated to the interaction of the Beagle Bone with the low level controller (soil, IMUs, motors) and with the users. The different functions are controlled by different components : 

\begin{enumerate}
 	\item Controllers : read sensors and send motor commands. Since this software pack is already used in two different exoskeletons (and a third one will be added at the beginning of 2017 with the integration of the Autonomyo), this project has already many controllers which can use different captors or control different motors. The \textit{WalkiBB} (controller used for the Twice exoskeleton shown at Cybathlon) and the HibSo (the current controller based on adaptive oscillators used to control the HibSo) are examples of such controllers. We will use the HibSo controller as the basis for the design of the HibSo Reflex Controller. 
	\item Interaction with the user : the \textit{RemoteControl} and Android components can be used to create user interface on the computer / android based smart phone respectively. In this project we implement our own version of the \textit{RemoteControl}.
	\item Logging, analysis : utilities composed by a set of Matlab script to interpret log files (conversion and viewing), by a joint simulator or by motorboard tester. 
\end{enumerate}

The Walki Software Pack can be downloaded on \url{https://git.epfl.ch/repo/walki-software.git} (send a email to romain.baud@epfl.ch to have access to it).

\subsubsection{Current controller}
\label{actualcontroller}
The current controller is based on the oscillator-based assistance developed by Renaud Ronsse~\cite{ronsse2011oscillator}.  The great feature about this controller is that the movement is considered like a sinusoidal movement. The algorithm reproduces this oscillation on the motor and adapt its features (like the amplitude or the frequency) to the user purpose.

% complete explication : http://ieeexplore.ieee.org/document/5975352/?arnumber=5975352 

\subsubsection{Libnmm library}

The library \textit{Libnmm}~\cite{libnmm} is developed and maintained by Florin Dzeladini. It reproduces - in c++ - and extends with CPG the "spinal network model of locomotion" proposed by H. Geyer~\cite{geyer2010muscle}, more information on the CPG model can be found in~\cite{dzeladini2014contribution}. The idea of the project is to turn the "spinal network model of locomotion" into an exoskeleton controller. The basic idea used in this project is to use the torque output from the model to drive the exoskeletons motors, see Section~\ref{controller_design}. One of the main challenges of this project was to turn the \textit{Libnmm} library into a library for the design of bio-inspired exoskeleton controller, which mainly involved its modification to make it real-time ready.  The target goal is to run the controller at 1Khz on the Beaglbone board (see Section~\ref{hardware}, the same rate than the one on which the model has been optimized \textit{Webots} model. This means that we want the controller to be able to run on the exoskeleton main board at a sufficiently fast and stable rate. The main considerations to apply to the code are :
\begin{enumerate}
	\item allocate all the memory at the initialization 
    \item optimize the code to make it run faster 
    \item (if the two last steps are not enough), disable unnecessary part of the code. Indeed since for now we use only the hip controller we can disable all muscles on knee and ankle. 
\end{enumerate}

\section{Results}

\subsection{Soil characterization}

\subsubsection{Step Detection}

The first tests of the step detection were done with a force plate, the ground truth, and the sole. We reproduce the protocol below, five times to be sure of our results.\\

Test protocol
\begin{enumerate}
  \item two right steps + two left steps
  \item five seconds waiting 
  \item normal walking (five right steps + five left steps)
  \item five seconds waiting
  \item fast walking (ten right steps + ten left steps)
  \item five seconds waiting
  \item two right steps + two left steps\\
\end{enumerate}

The picture below shows the right foot force (red) and the force on the plate (black)\\

\begin{figure}[H]
	\centering
	\includegraphics[width=12cm]{sum_force_cell_right_and_x_force_plate.jpg}
    \caption{Sum force cell right (red) and X force plate (black). We can see that the two curves are similar in term of forms but not in term of scale. It can be explained by the cell distribution on each foot.}
    \label{cellrightforceplate}
\end{figure}

\begin{figure}[H]
	\centering
	\includegraphics[width=12cm]{step_detect_PLATE.jpg}
    \caption{Step detection on the X plate. Each jump on the blue plot corresponds to one step. We detect with precision the 19 right step. It is the same thing on the left step.}
\end{figure}

\begin{figure}[H]
	\centering
	\includegraphics[width=10cm]{step_detect_RIGHT.jpg}
    \caption{Step detection on the sum force cell right. We detect with precision the 19 right step. But we have some false positive. }
\end{figure}

\begin{figure}[H]
	\centering
	\includegraphics[width=12cm]{false_positiv.jpg}
    \caption{Zoom on a false positive. The non-flat position of the foot can be interpreted as a swing mode for a really shot time. Notice this kind of false positive does not appear anymore in real tests with an adapted threshold.}
\end{figure}

\begin{figure}[H]
	\centering
	\includegraphics[width=10cm]{step_detect_LEFT.jpg}
    \caption{Step detection on the sum force cell left. Some results as for the right : all swing mod are detected but some false positive appears}
\end{figure}

\subsubsection{Weights measure}
\label{soilintegration}
The initial H. Geyer model needed precise information of weights under each foot. As we can see on the Figure~\ref{cellrightforceplate}, we have an important gap between the real force (and so the weight) given by the force plate and the force given by the cells. So these sensors cannot be used to get a precise weight information.

\subsection{HibSo Reflex Controller}

The HibSo Reflex Controller is the new controller developed from \textit{Walki Software Pack}. Its design followed the following steps : 
\begin{enumerate}
	\item optimizes for a walking model in \textit{Webots}
    \item generates an independent controller from \textit{Webots}
    \item validates it with inputs and outputs extracted from the optimized \textit{Webots} model
    \item optimizes the speed of code with tests on the HibSo board and profiling
    \item adapts the controller to use inputs from the soil and the motors
\end{enumerate}

\subsubsection{Simulation}
\label{simulation}

The first step of the controller design involved the use of a simulation software called \textit{Webots}. It is a development environment used for mobile robots simulation. \textit{Webots} is distributed by company "Cyberbotics"~\cite{webots}. We can verify With this software the model and test the first steps of the controller.
\medbreak
\textit{Webots} allows us to optimize all the variables of the model which allows a smooth movement.The goal is to have a fastest stabilization possible.
\medbreak
The goal is to have the fastest stabilization possible. After tests, we stabilize after five steps and the simulation walks very smoothly with an important stability.

\subsubsection{Design and optimization of the controller with dump input}

First, the controller use the input data files of \textit{Webots}. It use each line to compute the new torques and adds it to the data file output. After that, we compare the output file from \textit{Hibso Reflex} with the output file of \textit{Webots}. The goal is to have no difference between the two.
\medbreak
The \textit{Libnmm} library was updated with the new coefficients results from the simulation optimization and allows us to compare the \textit{HibsoReflex} code outputs and the \textit{Webots}  simulation outputs which are both extremely similar. 
\medbreak
The \textit{Libnmm} library needed some improvement because of the big obligation in terms of duration : modification/removal of some function or change of some declaration (more explanation below). We change exponential by a simpler look-up table, change cosinus and sinus by approximation, modification of the "event manager" (remove not useful function in our case like the noise simulation or the ECG interpretation) and replace the variable declared in double by declaration in floats.

\subsubsection{Soil integration}

The current controller does not have sole driver. The first thing to do is to recover sensors data. A sole is actually composed of 8 cells. The recovered value of each cell is a voltage value. In (\ref{conversion}), you can see the conversion formula.

\begin{equation}
	\label{conversion}
	F[N]=\frac{(289000*Uadc[V])}{(9900-Uadc[V]*7708}
\end{equation}

\medbreak
For the step detection algorithm, we sum the force given by all cells of one sole. If one force is lower than a threshold for a small duration (the max duration was set to one millisecond for the first tests), we supposed the foot swing. This method is applied to each foot.
\medbreak
Once determining whether the foot is raised or not, we update a value of the 'HiBSOReflexController' class named \verb|stance_or_swing|, can take three values (\verb|LEFT_STANCE|, \verb|RIGHT_STANCE|, \verb|BOTH_STANCE|) and this value is displayed as two buttons (one by foot) that changes color (red = stance, black = swing) on the \textit{RemoteControl} software.

\subsubsection{Controller's User interface}
The user interface design was done in parallel during the project.  
\medbreak

\begin{figure}[H]
\begin{minipage}[c]{.43\linewidth}
     \begin{center}
             \includegraphics[width=7.6cm]{remotecontrolenosaturation.jpg}
             \caption{Assistance without saturation}
             \label{nosaturation}
         \end{center}
   \end{minipage} \hfill
   \begin{minipage}[c]{.43\linewidth}
    \begin{center}
            \includegraphics[width=7.6cm]{remotecontrolewithsaturation.jpg}
			\caption{Assistance with saturation}
            \label{saturation}
        \end{center}
 \end{minipage}
\end{figure}

The main gain from this interface is to allow user control and see every aspect of the assistance (visual on figures \ref{saturation} and \ref{nosaturation}):
\begin{enumerate}
	\item Help activation: Enables or disables the assistance
    \item Right and Left: chose a specific help for each foot (in the case where the patient is not constrained in the same way for each foot) 
    \item Saturation: Gives the maximum permissive torque to begin the assistance safely.
    \item Step detection icon: allows to see if foots are stance or swing. It is more a control specification than an important feature for the patient.
    \item Torques Viewing: allows to see the value of compute torques. It is more a control specification than a important feature for the patient.								
\end{enumerate}

In comparison with the current controller, it can detect if the step is stance or swing (as required by the H.Geyer model), use the \textit{Libnmm} library (to compute new torques), create log files with the torques output and show new variables via \textit{RemoteControl}.
\medbreak
The \textit{RemoteControl} was updated with a \textit{Torques and Step} sheet. On this sheet, we can visualize the torques output and which feet is up (or down). We can also enable or disable the walking support and chose the strength of the help separately for each foot. \\

\subsection{Controller efficiency}

The controller has first been validated with dump input (extract from~\textit{Webots}) on the computer (intel core i7 processor) (to show that the computed torques are the same as the ones in~\textit{Webots}. The computational time was sufficiently small, we obtain a mean duration of \textbf{0.14 ms} whereas we must be less than \textbf{1 ms}.
\medbreak
On the board, the computational time had been increased by a factor 20, and reached almost 3ms. After the duration optimization, we got a good mean duration of \textbf{0.7-0.9 ms} (mean evolution on~\ref{withoutmotor} with some peaks at random moment.
The big problems arrived when we connected one motorboard. At this time, the duration time increased to \textbf{0.7-1.0 ms} with peaks to \textbf{1.7 ms}. The duration is even longer when we tested the controller on the exoskeleton which has two motorboards.

\begin{figure}[H]
\begin{minipage}[c]{.45\linewidth}
     \begin{center}
             \includegraphics[width=7cm]{mainloop_without.jpg}
             \label{withoutmotor}
             \caption{Compute mean (every 10 ms) duration of the main loop without the motorboard}
         \end{center}
         
   \end{minipage} \hfill
   \begin{minipage}[c]{.45\linewidth}
    \begin{center}
            \includegraphics[width=7cm]{mainloop_with.jpg}
            \label{withmotor}
            \caption{Compute mean (every 10 ms) duration of the main loop with the motorboard}
        \end{center}
 \end{minipage}
\end{figure}

The peaks and the longer duration on the device can be explained by the priority system of Linux. The \textit{HibsoReflex} controller works in parallel with the Linux system set up on the \textit{BeagleBone}. We want a very high frequency (1kHz) for the main loop and this can interferes with the Linux kernel and cause priority issues. \\
The increase duration with the motorboard is normal because the controller has more inputs to manage. But, in comparison with the old controller, the duration should not be so high. It means that the problem is coming from the interaction between the low level communication and the new controller. This is the reason why we could not test the controller on the device during this project. However, this is the very last step before the test of the controller and we are confident we will very soon solve this issue and do the first tests.

\section{Discussions and future work}

As discussed in~\ref{HAM}, we have voluntary removed the HAM influence of the torques computation. It could be interesting to activate it to see the influence on the patient and thus the importance of the HAM reflex loop.
\medbreak
Concerning step detection from cells, \textit{swing} is always detected but some false positive can happen during stance. Thus the sole needs to be flat. It is not very important at the beginning, we just need to be careful that the sole is really flat. In the future, we may have to change these sensors by sole with more cells on the sides. %ADDED
As the false positive happens before the stance ends, an other alternative could be to use a small filter combined with a threshold.
Concerning the use of the soil to get the weight information, the results show that the soil can not be used to reliably estimate the weight, but could still be used to get an estimate of the Center of pressure.
\medbreak
The computational time problem can be reduced in different ways. The first is to change the micro controller and use a more powerful one. %But this solution has a very high development time cost. 
This solution will be studied by Romain Baud. The second is to change the main loop frequency : we certainly do not need to stay at 1 kHz at each moment of the loop. We could try to optimize a gait at 2 Khz on Webots, if the locomotion emerges then we could run the controller at 2Khz with the obtained parameters.
%For example the torques computation needs this frequency but the PID regulation does not need so high degree of precision, we could use a Cascade controller in that case.
\label{discussions}
\medbreak
At this state of the project, after the correction of the last inaccuracies, tests on real patients start to be conceivable. A rigorous protocol can be put in place to test an important number of people and prove or disprove the efficiency of the controller.
\medbreak
For the next version of the controller, we may add more modules (table~\ref{modules}) than now. For example, with the addition of an IMU to the card, we could use the \textit{Trunk Stabilization} and with more sensors and actuators on the knee and the ankle, we could add the \textit{Push off} module and  increase the efficiency and precision of the existent \textit{Swing} and \textit{Weight Transfer} modules.

\newpage
\section{Conclusion}

This project intends to prove that H.Geyer model is very interesting and has a great potential for rehabilitation. Even if the first results do not validate the specification duration (compute loop < 1ms), the controller can be easily improved (as discussed in the~\ref{discussions}). Furthermore, we have important possibilities for the future. For example, we can extend the model to the entire leg or/and use CPG information to control the locomotion speed or adapt the step length. This kind of improvement will be extremely beneficial for the rehabilitation of injured peoples or for disabled persons.
\medbreak
In a more personal point of view, this project was very rewarding because it allowed me to try many different and interesting fields. Despite my lack of knowledge at the beginning, this project allowed me to consolidate my programming ability, to learn more about the exoskeleton theory and its stakes and discover many new features about the human body and in particular, leg and the role of muscles.  
The number of different models, the way to simulate and optimize each of them, how to test the efficiency and the patients proximity (about how they could be helped) motivates me to keep working in this field in the future.

\newpage
\bibliographystyle{plain}
\bibliography{biblio}

\end{document}
